
% Default to the notebook output style

    


% Inherit from the specified cell style.




    
\documentclass[twocolumn]{article}

    \usepackage{wrapfig}
    \usepackage[font={small}]{caption}
    \usepackage{graphicx} % Used to insert images
    \usepackage{adjustbox} % Used to constrain images to a maximum size 
    \usepackage{color} % Allow colors to be defined
    \usepackage{enumerate} % Needed for markdown enumerations to work
    \usepackage{geometry} % Used to adjust the document margins
    \usepackage{amsmath} % Equations
    \usepackage{amssymb} % Equations
    \usepackage[mathletters]{ucs} % Extended unicode (utf-8) support
    \usepackage[utf8x]{inputenc} % Allow utf-8 characters in the tex document
    \usepackage{fancyvrb} % verbatim replacement that allows latex
    \usepackage{grffile} % extends the file name processing of package graphics 
                         % to support a larger range 
    % The hyperref package gives us a pdf with properly built
    % internal navigation ('pdf bookmarks' for the table of contents,
    % internal cross-reference links, web links for URLs, etc.)
    \usepackage{hyperref}
    \usepackage{longtable} % longtable support required by pandoc >1.10
    \usepackage[russian]{babel}
	\usepackage{dblfloatfix}
	\usepackage{subcaption}
	\usepackage{float}
    \setlength{\parskip}{\baselineskip}%
	\setlength{\parindent}{0pt}%
    
    \definecolor{orange}{cmyk}{0,0.4,0.8,0.2}
    \definecolor{darkorange}{rgb}{.71,0.21,0.01}
    \definecolor{darkgreen}{rgb}{.12,.54,.11}
    \definecolor{myteal}{rgb}{.26, .44, .56}
    \definecolor{gray}{gray}{0.45}
    \definecolor{lightgray}{gray}{.95}
    \definecolor{mediumgray}{gray}{.8}
    \definecolor{inputbackground}{rgb}{.95, .95, .85}
    \definecolor{outputbackground}{rgb}{.95, .95, .95}
    \definecolor{traceback}{rgb}{1, .95, .95}
    % ansi colors
    \definecolor{red}{rgb}{.6,0,0}
    \definecolor{green}{rgb}{0,.65,0}
    \definecolor{brown}{rgb}{0.6,0.6,0}
    \definecolor{blue}{rgb}{0,.145,.698}
    \definecolor{purple}{rgb}{.698,.145,.698}
    \definecolor{cyan}{rgb}{0,.698,.698}
    \definecolor{lightgray}{gray}{0.5}
    
    % bright ansi colors
    \definecolor{darkgray}{gray}{0.25}
    \definecolor{lightred}{rgb}{1.0,0.39,0.28}
    \definecolor{lightgreen}{rgb}{0.48,0.99,0.0}
    \definecolor{lightblue}{rgb}{0.53,0.81,0.92}
    \definecolor{lightpurple}{rgb}{0.87,0.63,0.87}
    \definecolor{lightcyan}{rgb}{0.5,1.0,0.83}
    
    % commands and environments needed by pandoc snippets
    % extracted from the output of `pandoc -s`
    \DefineVerbatimEnvironment{Highlighting}{Verbatim}{commandchars=\\\{\}}
    % Add ',fontsize=\small' for more characters per line
    \newenvironment{Shaded}{}{}
    \newcommand{\KeywordTok}[1]{\textcolor[rgb]{0.00,0.44,0.13}{\textbf{{#1}}}}
    \newcommand{\DataTypeTok}[1]{\textcolor[rgb]{0.56,0.13,0.00}{{#1}}}
    \newcommand{\DecValTok}[1]{\textcolor[rgb]{0.25,0.63,0.44}{{#1}}}
    \newcommand{\BaseNTok}[1]{\textcolor[rgb]{0.25,0.63,0.44}{{#1}}}
    \newcommand{\FloatTok}[1]{\textcolor[rgb]{0.25,0.63,0.44}{{#1}}}
    \newcommand{\CharTok}[1]{\textcolor[rgb]{0.25,0.44,0.63}{{#1}}}
    \newcommand{\StringTok}[1]{\textcolor[rgb]{0.25,0.44,0.63}{{#1}}}
    \newcommand{\CommentTok}[1]{\textcolor[rgb]{0.38,0.63,0.69}{\textit{{#1}}}}
    \newcommand{\OtherTok}[1]{\textcolor[rgb]{0.00,0.44,0.13}{{#1}}}
    \newcommand{\AlertTok}[1]{\textcolor[rgb]{1.00,0.00,0.00}{\textbf{{#1}}}}
    \newcommand{\FunctionTok}[1]{\textcolor[rgb]{0.02,0.16,0.49}{{#1}}}
    \newcommand{\RegionMarkerTok}[1]{{#1}}
    \newcommand{\ErrorTok}[1]{\textcolor[rgb]{1.00,0.00,0.00}{\textbf{{#1}}}}
    \newcommand{\NormalTok}[1]{{#1}}
    
    % Define a nice break command that doesn't care if a line doesn't already
    % exist.
    \def\br{\hspace*{\fill} \\* }
    % Math Jax compatability definitions
    \def\gt{>}
    \def\lt{<}
    % Document parameters
    \title{Исследование наночастиц платины, нанесенных на дисперсный углерод, методом рентгеновской дифрактометрии}
    \author{Кинзина Э. Д., Федоров Г. П., Ушаков А. Е., Храмцов И. А., Ушков А. А.}
    
    
    

    % Pygments definitions
    
\makeatletter
\def\PY@reset{\let\PY@it=\relax \let\PY@bf=\relax%
    \let\PY@ul=\relax \let\PY@tc=\relax%
    \let\PY@bc=\relax \let\PY@ff=\relax}
\def\PY@tok#1{\csname PY@tok@#1\endcsname}
\def\PY@toks#1+{\ifx\relax#1\empty\else%
    \PY@tok{#1}\expandafter\PY@toks\fi}
\def\PY@do#1{\PY@bc{\PY@tc{\PY@ul{%
    \PY@it{\PY@bf{\PY@ff{#1}}}}}}}
\def\PY#1#2{\PY@reset\PY@toks#1+\relax+\PY@do{#2}}

\expandafter\def\csname PY@tok@nl\endcsname{\def\PY@tc##1{\textcolor[rgb]{0.63,0.63,0.00}{##1}}}
\expandafter\def\csname PY@tok@nn\endcsname{\let\PY@bf=\textbf\def\PY@tc##1{\textcolor[rgb]{0.00,0.00,1.00}{##1}}}
\expandafter\def\csname PY@tok@sx\endcsname{\def\PY@tc##1{\textcolor[rgb]{0.00,0.50,0.00}{##1}}}
\expandafter\def\csname PY@tok@ni\endcsname{\let\PY@bf=\textbf\def\PY@tc##1{\textcolor[rgb]{0.60,0.60,0.60}{##1}}}
\expandafter\def\csname PY@tok@ss\endcsname{\def\PY@tc##1{\textcolor[rgb]{0.10,0.09,0.49}{##1}}}
\expandafter\def\csname PY@tok@ne\endcsname{\let\PY@bf=\textbf\def\PY@tc##1{\textcolor[rgb]{0.82,0.25,0.23}{##1}}}
\expandafter\def\csname PY@tok@nf\endcsname{\def\PY@tc##1{\textcolor[rgb]{0.00,0.00,1.00}{##1}}}
\expandafter\def\csname PY@tok@kd\endcsname{\let\PY@bf=\textbf\def\PY@tc##1{\textcolor[rgb]{0.00,0.50,0.00}{##1}}}
\expandafter\def\csname PY@tok@na\endcsname{\def\PY@tc##1{\textcolor[rgb]{0.49,0.56,0.16}{##1}}}
\expandafter\def\csname PY@tok@nb\endcsname{\def\PY@tc##1{\textcolor[rgb]{0.00,0.50,0.00}{##1}}}
\expandafter\def\csname PY@tok@nc\endcsname{\let\PY@bf=\textbf\def\PY@tc##1{\textcolor[rgb]{0.00,0.00,1.00}{##1}}}
\expandafter\def\csname PY@tok@k\endcsname{\let\PY@bf=\textbf\def\PY@tc##1{\textcolor[rgb]{0.00,0.50,0.00}{##1}}}
\expandafter\def\csname PY@tok@nt\endcsname{\let\PY@bf=\textbf\def\PY@tc##1{\textcolor[rgb]{0.00,0.50,0.00}{##1}}}
\expandafter\def\csname PY@tok@si\endcsname{\let\PY@bf=\textbf\def\PY@tc##1{\textcolor[rgb]{0.73,0.40,0.53}{##1}}}
\expandafter\def\csname PY@tok@no\endcsname{\def\PY@tc##1{\textcolor[rgb]{0.53,0.00,0.00}{##1}}}
\expandafter\def\csname PY@tok@gi\endcsname{\def\PY@tc##1{\textcolor[rgb]{0.00,0.63,0.00}{##1}}}
\expandafter\def\csname PY@tok@sc\endcsname{\def\PY@tc##1{\textcolor[rgb]{0.73,0.13,0.13}{##1}}}
\expandafter\def\csname PY@tok@sb\endcsname{\def\PY@tc##1{\textcolor[rgb]{0.73,0.13,0.13}{##1}}}
\expandafter\def\csname PY@tok@nv\endcsname{\def\PY@tc##1{\textcolor[rgb]{0.10,0.09,0.49}{##1}}}
\expandafter\def\csname PY@tok@se\endcsname{\let\PY@bf=\textbf\def\PY@tc##1{\textcolor[rgb]{0.73,0.40,0.13}{##1}}}
\expandafter\def\csname PY@tok@sd\endcsname{\let\PY@it=\textit\def\PY@tc##1{\textcolor[rgb]{0.73,0.13,0.13}{##1}}}
\expandafter\def\csname PY@tok@err\endcsname{\def\PY@bc##1{\setlength{\fboxsep}{0pt}\fcolorbox[rgb]{1.00,0.00,0.00}{1,1,1}{\strut ##1}}}
\expandafter\def\csname PY@tok@vi\endcsname{\def\PY@tc##1{\textcolor[rgb]{0.10,0.09,0.49}{##1}}}
\expandafter\def\csname PY@tok@nd\endcsname{\def\PY@tc##1{\textcolor[rgb]{0.67,0.13,1.00}{##1}}}
\expandafter\def\csname PY@tok@gt\endcsname{\def\PY@tc##1{\textcolor[rgb]{0.00,0.27,0.87}{##1}}}
\expandafter\def\csname PY@tok@sr\endcsname{\def\PY@tc##1{\textcolor[rgb]{0.73,0.40,0.53}{##1}}}
\expandafter\def\csname PY@tok@bp\endcsname{\def\PY@tc##1{\textcolor[rgb]{0.00,0.50,0.00}{##1}}}
\expandafter\def\csname PY@tok@c1\endcsname{\let\PY@it=\textit\def\PY@tc##1{\textcolor[rgb]{0.25,0.50,0.50}{##1}}}
\expandafter\def\csname PY@tok@s1\endcsname{\def\PY@tc##1{\textcolor[rgb]{0.73,0.13,0.13}{##1}}}
\expandafter\def\csname PY@tok@kr\endcsname{\let\PY@bf=\textbf\def\PY@tc##1{\textcolor[rgb]{0.00,0.50,0.00}{##1}}}
\expandafter\def\csname PY@tok@mf\endcsname{\def\PY@tc##1{\textcolor[rgb]{0.40,0.40,0.40}{##1}}}
\expandafter\def\csname PY@tok@il\endcsname{\def\PY@tc##1{\textcolor[rgb]{0.40,0.40,0.40}{##1}}}
\expandafter\def\csname PY@tok@kt\endcsname{\def\PY@tc##1{\textcolor[rgb]{0.69,0.00,0.25}{##1}}}
\expandafter\def\csname PY@tok@s2\endcsname{\def\PY@tc##1{\textcolor[rgb]{0.73,0.13,0.13}{##1}}}
\expandafter\def\csname PY@tok@mo\endcsname{\def\PY@tc##1{\textcolor[rgb]{0.40,0.40,0.40}{##1}}}
\expandafter\def\csname PY@tok@mi\endcsname{\def\PY@tc##1{\textcolor[rgb]{0.40,0.40,0.40}{##1}}}
\expandafter\def\csname PY@tok@mh\endcsname{\def\PY@tc##1{\textcolor[rgb]{0.40,0.40,0.40}{##1}}}
\expandafter\def\csname PY@tok@kc\endcsname{\let\PY@bf=\textbf\def\PY@tc##1{\textcolor[rgb]{0.00,0.50,0.00}{##1}}}
\expandafter\def\csname PY@tok@vg\endcsname{\def\PY@tc##1{\textcolor[rgb]{0.10,0.09,0.49}{##1}}}
\expandafter\def\csname PY@tok@vc\endcsname{\def\PY@tc##1{\textcolor[rgb]{0.10,0.09,0.49}{##1}}}
\expandafter\def\csname PY@tok@ow\endcsname{\let\PY@bf=\textbf\def\PY@tc##1{\textcolor[rgb]{0.67,0.13,1.00}{##1}}}
\expandafter\def\csname PY@tok@kn\endcsname{\let\PY@bf=\textbf\def\PY@tc##1{\textcolor[rgb]{0.00,0.50,0.00}{##1}}}
\expandafter\def\csname PY@tok@ge\endcsname{\let\PY@it=\textit}
\expandafter\def\csname PY@tok@gd\endcsname{\def\PY@tc##1{\textcolor[rgb]{0.63,0.00,0.00}{##1}}}
\expandafter\def\csname PY@tok@m\endcsname{\def\PY@tc##1{\textcolor[rgb]{0.40,0.40,0.40}{##1}}}
\expandafter\def\csname PY@tok@o\endcsname{\def\PY@tc##1{\textcolor[rgb]{0.40,0.40,0.40}{##1}}}
\expandafter\def\csname PY@tok@cm\endcsname{\let\PY@it=\textit\def\PY@tc##1{\textcolor[rgb]{0.25,0.50,0.50}{##1}}}
\expandafter\def\csname PY@tok@go\endcsname{\def\PY@tc##1{\textcolor[rgb]{0.53,0.53,0.53}{##1}}}
\expandafter\def\csname PY@tok@c\endcsname{\let\PY@it=\textit\def\PY@tc##1{\textcolor[rgb]{0.25,0.50,0.50}{##1}}}
\expandafter\def\csname PY@tok@kp\endcsname{\def\PY@tc##1{\textcolor[rgb]{0.00,0.50,0.00}{##1}}}
\expandafter\def\csname PY@tok@gh\endcsname{\let\PY@bf=\textbf\def\PY@tc##1{\textcolor[rgb]{0.00,0.00,0.50}{##1}}}
\expandafter\def\csname PY@tok@gu\endcsname{\let\PY@bf=\textbf\def\PY@tc##1{\textcolor[rgb]{0.50,0.00,0.50}{##1}}}
\expandafter\def\csname PY@tok@sh\endcsname{\def\PY@tc##1{\textcolor[rgb]{0.73,0.13,0.13}{##1}}}
\expandafter\def\csname PY@tok@gs\endcsname{\let\PY@bf=\textbf}
\expandafter\def\csname PY@tok@gr\endcsname{\def\PY@tc##1{\textcolor[rgb]{1.00,0.00,0.00}{##1}}}
\expandafter\def\csname PY@tok@gp\endcsname{\let\PY@bf=\textbf\def\PY@tc##1{\textcolor[rgb]{0.00,0.00,0.50}{##1}}}
\expandafter\def\csname PY@tok@cs\endcsname{\let\PY@it=\textit\def\PY@tc##1{\textcolor[rgb]{0.25,0.50,0.50}{##1}}}
\expandafter\def\csname PY@tok@s\endcsname{\def\PY@tc##1{\textcolor[rgb]{0.73,0.13,0.13}{##1}}}
\expandafter\def\csname PY@tok@cp\endcsname{\def\PY@tc##1{\textcolor[rgb]{0.74,0.48,0.00}{##1}}}
\expandafter\def\csname PY@tok@w\endcsname{\def\PY@tc##1{\textcolor[rgb]{0.73,0.73,0.73}{##1}}}

\def\PYZbs{\char`\\}
\def\PYZus{\char`\_}
\def\PYZob{\char`\{}
\def\PYZcb{\char`\}}
\def\PYZca{\char`\^}
\def\PYZam{\char`\&}
\def\PYZlt{\char`\<}
\def\PYZgt{\char`\>}
\def\PYZsh{\char`\#}
\def\PYZpc{\char`\%}
\def\PYZdl{\char`\$}
\def\PYZhy{\char`\-}
\def\PYZsq{\char`\'}
\def\PYZdq{\char`\"}
\def\PYZti{\char`\~}
% for compatibility with earlier versions
\def\PYZat{@}
\def\PYZlb{[}
\def\PYZrb{]}
\makeatother


    % Exact colors from NB
    \definecolor{incolor}{rgb}{0.0, 0.0, 0.5}
    \definecolor{outcolor}{rgb}{0.545, 0.0, 0.0}



    
    % Prevent overflowing lines due to hard-to-break entities
    \sloppy 
    % Setup hyperref package
    \hypersetup{
      breaklinks=true,  % so long urls are correctly broken across lines
      colorlinks=true,
      urlcolor=blue,
      linkcolor=darkorange,
      citecolor=darkgreen,
      }
    % Slightly bigger margins than the latex defaults
    
    \geometry{verbose,tmargin=1in,bmargin=1in,lmargin=1in,rmargin=1in}
    
    \raggedbottom

    \begin{document}

        
            \maketitle
       

    
\section*{Содержание}
\begin{enumerate}
\def\labelenumi{\arabic{enumi}.}
\item
  Принципы работы квантового генератора
\item
  Неодимовый лазер
    \begin{enumerate}[{2.}1]

	\item
  Принципы юстировки
	\item
  Спектр и форма импульса лампы-вспышки
	\item
  Энергия импульса
	\item
  Спектр
	\item
  Форма импульсов
  \end{enumerate}
\item
  Неодимовый лазер с модулированной добротностью
    \begin{enumerate}[{3.}1]

	\item
  Нелинейные поглотители
	\item
  Энергия импульса
	\item
  Форма импульсов
	\item
  Пробой воздуха
  \end{enumerate}
\item
  Вторая гармоника
    \begin{enumerate}[{4.}1]

	\item
  Теоретическое введение
	\item
  Результаты
  \end{enumerate}
\item
  Приложения
    \begin{enumerate}[{5.}1]

	\item
  Осциллограммы, не вошедшие в основной отчет
  
	\item
  Спектры, не вошедшие в основной отчет
  \end{enumerate}
\end{enumerate}
\section{Принципы работы квантового генератора}Лазерный излучатель, служащий для генерации электромагнитного излучения
оптического диапазона с уникальными свойствами, структурно состоит из
следующих основных элементов: активной среды, источника накачки и
резонатора.$\newline$ Активная среда − это вещество, в котором может
быть создана инверсная населённость энергетических уровней, т.е.
достигнуто такое состояние, когда число атомов, находящихся на верхнем
``рабочем'' энергетическом уровне, превышает число атомов, находящихся
на нижнем ``рабочем'' энергетическом уровне. По типу активной среды
лазеры подразделяются на твердотельные, газовые, полупроводниковые,
жидкостные и др. На практике активную среду твердотельных лазеров часто
также называют активным элементом.$\newline$ Поставщиком энергии для
достижения состояния инверсной населённости служит источник накачки, в
качестве которого может выступать, например: лампа−вспышка, газовый
разряд, инжекция носителей тока в полупроводниковых p−n переходах,
тепловой способ, химическая реакция и др.$\newline$ Важнейшей и
неотъемлемой частью любого лазера является резонатор − система,
состоящая, как правило, из двух отражающих поверхностей, между которыми
располагается активная среда.$\newline$ Отражающие поверхности могут
представлять собой зеркала различной формы (плоские, сферические,
параболические и др.), грани призм полного внутреннего отражения или
границы раздела сред с различными показателями преломления.$\newline$
Зеркала лазера чаще всего формируются путём нанесения многослойных
отражающих диэлектрических покрытий на одну из отполированных по
специальной технологии поверхностей основы зеркала − на так называемую
подложку. На другую поверхность подложки зеркала либо наносят
просветляющее покрытие, либо её оставляют без покрытия. Поверхность
зеркала с отражающим покрытием называется ``рабочей'', одно из зеркал
резонатора, полностью отражающее свет, − ``глухим'', а зеркало, частично
пропускающее излучение, − выходным. Основным назначением оптического
резонатора является создание условий, при которых возникающее в активной
среде излучение, многократно проходя через её структуру, усиливается до
уровня превышения имеющихся потерь. Следовательно, резонатор
осуществляет положительную обратную связь. Другим его немаловажным
свойством является спектральная селекция излучения.$\newline$ Самым
простым и наиболее распространённым видом резонатора является система из
двух плоских зеркал, обращённых друг к другу отражающими поверхностями,
называемая эталоном Фабри−Перо. Под юстировкой системы в общем случае
понимают совокупность операций по приведению её элементов в состояние,
обеспечивающее правильное функционирование системы.$\newline$ Юстировка
оптических систем заключается в регулировании взаимного расположения
оптических деталей (линз, призм, зеркал и т.п.) с целью их центрирования
и обеспечения наилучшего качества изображения. В съюстированном
положении оптические детали закрепляются винтами, штифтами либо
склеиваются.$\newline$ Физический смысл процесса юстировки лазера
состоит в нахождении такого расположения его оптических элементов
(активной среды, зеркал резонатора и т.п.) друг относительно друга, при
котором потери излучения минимальны.$\newline$ На практике юстировку
разделяют на ``холодную'' (или грубую) и ``горячую'' (или
тонкую).$\newline$ В первом случае широкое практическое применение
получил метод так называемого оптического рычага, для реализации
которого необходим вспомогательный юстировочный лазер − низкоинтенсивный
лазер, генерирующий излучение в видимом диапазоне. Источник же накачки
юстируемого лазера в этом случае не включается.$\newline$ ``Горячая''
юстировка осуществляется уже при непосредственном включении источника
накачки юстируемого лазера, что позволяет учесть термооптические
особенности настраиваемой системы. Данный вид юстировки производится с
использованием специальных средств, в качестве которых обычно выступают:
визуализаторы и регистраторы излучения (например, копировальная бумага,
фотобумага и др.), фотоприёмник, измеритель энергии/мощности лазерного
излучения и др.$\newline$ Для оценки степени чувствительности лазерного
резонатора к разъюстировке служит разъюстировочная характеристика,
представляющая собой зависимость параметров лазерной генерации от угла
отклонения одного из зеркал резонатора относительно съюстированного
положения.$\newline$Под режимом свободной генерации лазера понимают такой режим, при котором
отсутствует какое−либо целенаправленное управление параметрами или
элементами лазерного излучателя в процессе генерации. Так как в режиме
свободной генерации отсутствуют дополнительные потери излучения в
резонаторе на элементах управления, то лазеры имеют здесь наибольшую
энергию импульса генерации. Соответственно, в этом режиме лазер обладает
и наибольшим коэффициентом полезного действия − КПД. 
\section{Неодимовый лазер}
\subsection{Принципы юстировки}Для получения максимальной мощности пучка при минимальной расходимости
были использован следующий порядок действий:\subsubsection{\emph{Холодная юстировка}}\begin{enumerate}
\def\labelenumi{\arabic{enumi}.}
\item
  Получить разъяснения от преподавателя об особенностях выполнения
  данного задания.
\item
  Включить вспомогательный юстировочный лазер \textbf{1}. Его луч должен
  пройти через отверстие в юстировочной диафрагме \textbf{2}.
\item
  Установить на пути следования луча от вспомогательного юстировочного
  лазера \textbf{1} активный элемент \textbf{6} юстируемого лазера таким
  образом, чтобы луч от вспомогательного лазера \textbf{1} проходил
  вдоль его оси через центр. Позиционирование проводить с помощью
  юстировочных винтов оптической подвижки, в которой закреплён
  юстируемый активный элемент \textbf{6}. Контроль за прохождением луча
  от вспомогательного юстировочного лазера \textbf{1} осуществлять
  визуально.
\item
  Посредством юстировочных винтов подвижки активного элемента
  \textbf{6}, ответственных за угловые перемещения объекта, добиться
  наиболее точного совпадения блика (светового отражения) от входного
  торца активного элемента \textbf{6} с отверстием диафрагмы \textbf{2}.
\item
  На выходе из активного элемента \textbf{6} проверить ещё раз качество
  излучения от вспомогательного юстировочного лазера \textbf{1}.
\item
  Найти на экране \textbf{8} блик от входного торца активного элемента
  \textbf{6}.
\item
  Зафиксировать на экране \textbf{8} местоположение блика от входного
  торца активного элемента \textbf{6} (см. рис. 1.1).
\item
  Не сбивая юстировки активного элемента \textbf{6}, установить на пути
  следования луча от вспомогательного юстировочного лазера \textbf{1}
  выходное зеркало \textbf{7} резонатора таким образом, чтобы данный луч
  проходил примерно через его центр. Подобная установка осуществляется с
  помощью юстировочных винтов оптической подвижки, ответственных за
  линейные перемещения объекта.
\item
  Наблюдая за экраном \textbf{8}, посредством юстировочных винтов
  подвижки выходного зеркала \textbf{7}, ответственных за угловые
  перемещения объекта, совместить блик от его поверхности с
  местоположением блика от входного торца активного элемента \textbf{6}
  (см. рис. 1.1).
\item
  Повторить все действия из п. 8 и п. 9 для ``глухого'' зеркала
  \textbf{5} резонатора.
\end{enumerate}\subsubsection{Горячая юстировка}\begin{enumerate}
\def\labelenumi{\arabic{enumi}.}
\itemsep1pt\parskip0pt\parsep0pt
\item
  Получить разъяснения от преподавателя об особенностях выполнения
  данного задания.
\item
  Установить на выходе из съюстированного в результате выполнения
  предыдущего задания лазерного излучателя фрагмент фотобумаги −
  объект−мишень \textbf{9}. ВНИМАНИЕ! Расстояние от выходного зеркала до
  поверхности мишени должно быть не менее 10 см.
\item
  При участии (!) преподавателя включить источник накачки (блок питания)
  юстируемого лазера.
\item
  По виду ожога, полученного на фотобумаге, оценить качество выполненной
  ранее ``холодной'' юстировки.
\item
  Дать объяснение полученному результату.
\item
  Продолжая использовать фотобумагу, путём незначительного вращения
  юстировочных винтов подвижки активного элемента \textbf{6}, отвечающих
  за угловые перемещения объекта, провести его доюстировку. ВНИМАНИЕ!
  Данные действия проводятся лишь при обязательном (!) присутствии
  преподавателя. Юстировка осуществляется при фиксированном значении
  энергии накачки блока питания.
\item
  По аналогии с предыдущим пунктом провести доюстировку выходного
  \textbf{7} и ``глухого'' \textbf{5} зеркал резонатора.
\end{enumerate}

 
         
    \begin{center}
    \includegraphics[width=0.5\textwidth]{LEMPH Report_files/LEMPH Report_14_0.png}
    \par
    \end{center}
               
        
    
\subsection{Спектр и форма импульса люминесценции}Ипользуя работающий на ПЗС-матрице мини-спектрометр FSD-9, мы получили
следующий спектр рассеянного излучения лампы-вспышки при напряжении на
питающем конденсаторе в 1.5 кВ


    
        
    \begin{center}
    \includegraphics[width=0.5\textwidth]{LEMPH Report_files/LEMPH Report_17_0.jpeg}
    \par
    \end{center}
           
            
Подавая сигнал с фотодиода (при освещении его рассеянным светом от света
вспышки) на вход цифрового осциллографа, получили следующую
осциллограмму, форму импульса во времени, опять же для 1.5 кВ:

   
    \begin{center}
    \includegraphics[width=0.5\textwidth]{LEMPH Report_files/LEMPH Report_19_0.jpeg}
    \par
    \end{center}
    
          
        
    
\subsection{Энергия импульса}Было проведено исследование работы лазера при различных энергиях
накачки, в результате получили следующий график зависимости интегральной
энергии импульса в зависимости от энергии накачки:

        
    \begin{center}
    \includegraphics[width=0.5\textwidth]{LEMPH Report_files/LEMPH Report_22_0.png}
    \par
    \end{center}
    
            
        
    
По графику видим, что пороговое значение энергии накачки составляет 14.9 Дж.
\subsection{Спектр}Спектр рассеянного излучения лазера на неодиме. Отчетливо видна длина
волны, на которой происходит генерация - 10600 $\mathring{A}$:

  
    
     
    \begin{center}
    \includegraphics[width=0.5\textwidth]{LEMPH Report_files/LEMPH Report_26_0.jpeg}
    \par
    \end{center}
    
            
        
    
\subsection{Форма импульса}Подавая сигнал с фотодиода (при освещении его рассеянным светом от
лазерного луча) на вход цифрового осциллографа, получили следующую
осциллограмму:

    
      
    \begin{center}
    \includegraphics[width=0.5\textwidth]{LEMPH Report_files/LEMPH Report_29_0.jpeg}
    \par
    \end{center}
    
         
            
        
    
\section{Неодимовый лазер с модулированной добротностью}\subsection{Нелинейные поглотители}Режим модуляции добротности твердотельного лазера служит для получения
мощного импульса малой длительности (1÷100 нс) или последовательности
таких импульсов. От режима свободной генерации он отличается тем, что
при нём первоначально с помощью внутрирезонаторного затвора
устанавливается малая добротность резонатора (большой уровень потерь).
Поскольку условия для возникновения генерации в лазере в этом случае не
выполняются, то под действием источника накачки в активном элементе
происходит значительное увеличение количества атомов на верхнем
``рабочем'' уровне. Если же теперь в некоторый момент времени быстро
переключить внутрирезонаторный затвор (т.е. увеличить добротность
резонатора), то коэффициент усиления излучения в резонаторе будет
значительно превышать остаточный уровень потерь. Именно это условие и
отличает работу лазера в режиме модулированной добротности от лазера,
работающего в режиме свободной генерации. Большое начальное значение
коэффициента усиления в режиме модулированной добротности по сравнению с
режимом свободной генерации приводит к уменьшению времени развития
импульса излучения, сокращению его длительности и увеличению мощности.
Энергия же одиночного импульса в режиме модулированной добротности из−за
присутствия в резонаторе источника дополнительных потерь
(внутрирезонаторного затвора), как правило, меньше, чем в режиме
свободной генерации.
\subsection{Энергия импульса}Для получения короткого мощного импульса использовались нелинейные
поглотители:

 
    \begin{center}
    \includegraphics[width=0.5\textwidth]{LEMPH Report_files/LEMPH Report_35_0.png}
    \par
    \end{center}
    

\newpage
    
\subsection{Форма импульсов}

    
       
    \begin{center}
    \includegraphics[width=0.5\textwidth]{LEMPH Report_files/LEMPH Report_37_0.jpeg}
    \par
    \end{center}
    
          \newpage  
        
    
\subsection{Пробой воздуха}Полученной мощности хватило для лазерного пробоя воздуха:

    % Make sure that atleast 4 lines are below the HR

    
       
    \begin{center}
    \includegraphics[width=0.5\textwidth]{LEMPH Report_files/LEMPH Report_40_0.jpeg}
    \par
    \end{center}
    
        
        
    
Удалось так же снять спектр вспышки, которая видна на рисунке. Он имеет
следующий вид:

    % Make sure that atleast 4 lines are below the HR

    
      
    \begin{center}
    \includegraphics[width=0.5\textwidth]{LEMPH Report_files/LEMPH Report_42_0.jpeg}
    \par
    \end{center}
                
        
    
\section{Вторая гармоника}
\subsection{Теоретическое введение}С точки зрения молекулярной теории явления преломления и отражения света
рассматриваются как результат интерференции падающей и вторичных волн,
испускаемых молекулами среды благодаря вынужденным колебаниям зарядов,
индуцированным падающей волной.

В рамках классического подхода распространение света в среде описывается
уравнениями Максвелла, дополненными материальными уравнениями. Если эти
уравнения линейны, то, согласно их решениям, световые волны с разными
характеристиками (например с разными частотами) распространяются в среде
независимо друг от друга, т. е. выполняется принцип суперпозиции. Такая
картина соответствует линейной оптике.

В классической линейной оптике предполагается, что индуцированная
поляризация среды $\vec{p}$ линейно зависит от напряженности
электрического поля волны $\vec{E}$ :\begin{align}
\vec{p} = \varepsilon_0 \chi_1 \vec{E}
\end{align}Здесь $\varepsilon_0$ -- электрическая постоянная; $\chi_1$ -- линейная
восприимчивость (поляризуемость) среды. Согласно (1), вынужденные
колебания зарядов совершаются с частотой внешнего поля, вследствие чего
падающая, отраженная и преломленная волны имеют одну и ту же частоту.
Выражение (1) применимо к изотропным средам, для которых величина
$\chi_1$ является скаляром. Если же среда анизотропна, то направления
векторов поляризации среды $\vec{p}$ и напряженности поля волны
$\vec{E}$ , вообще говоря, не совпадают. В этом случае линейная
восприимчивость среды является уже не скалярной, а тензорной величиной.
Например, в направлении $i$ кристалла составляющая поляризации $p_i$
будет выражаться через три составляющих поля $E_i$ :\begin{align}
p_i = \sum \varepsilon_0 \chi_{ij}E_{j}
\end{align}Соотношения (1) и (2) справедливы лишь при малых напряженнoстях $E$,
которые значительно ниже значений, характерных для внутриатомных
электрических полей $E_a\  (E_a \sim 10^{10} - 10^{11} V/m)$. Для световых
полей, создаваемых обычными нелазерными (тепловыми и люминесцентными)
источниками, напряженность ${E}$ не превышает $10^4-10^5 V/m$. Поэтому
линейная зависимость между ${p}$ и ${E}$ практически не нарушается.
Ситуация резко изменилась с появлением лазеров, позволивших получать
световые поля с напряженностью до $10^9 - 10^{10} V/m$, сравнимые с
внутриатомными полями. В этих условиях зависимость $P( E )$ приобретает
нелинейный характер. Поляризация изотропного диэлектрика в сильном поле
может быть представлена в виде ряда, содержащего нелинейные члены:\begin{align}
p = \varepsilon_0\chi_1E+\varepsilon_0\chi_2E^2+\varepsilon_0\chi_3E^3...,
\end{align}где $\chi_2$ , $\chi_3$ -- нелинейные восприимчивости первого, второго и т. д.
порядков, определяемые свойствами среды и не зависящие от $ E $. Отношение каждого последующего члена в правой части (3) к предыдущему имеет
значение порядка $E/E_a$ , поэтому для нелазерных источников с
$E << E_a$ все нелинейные слагаемые в разложении (3) пренебрежимо
малы. В сильных (лазерных) полях отклик атомного (молекулярного)
осциллятора на гармоническое воздействие оказывается негармоническим.
Другими словами, возникает возможность переизлучения не только на
частоте вынуждающего излучения $\omega$ , но и на кратных частотах $2 \omega$ ,
$3 \omega$ и т. д., т. е. генерации высших гармоник света. Пусть на среду падает световая волна частоты $\omega$:\begin{align}
E_{\omega} = E_{0\omega}cos(\omega t - k_\omega z)
\end{align}Воспользуемся выражением (3) для поляризации среды, сохранив в нем
только линейный и квадратичный члены (нелинейные слагаемые более высоких
порядков отвечают за генерацию третьей и более высоких гармоник света).
Нелинейная поляризация, связанная с квадратичным членом в (3), при
подстановке $Е$ из формулы (4) дает\begin{align}
\varepsilon_0\chi_2E^2 = \frac{1}{2}\varepsilon_0 \chi_2 E^2_{0\omega}\left\{1+cos\left[2\left(\omega t - k_\omega z\right)\right]\right\}
\end{align}Слагаемое $\varepsilon_0\chi_2 E^2_{0 \omega}/2$ в (5) соответствует постоянной
поляризации среды в поле мощной волны основной частоты $\omega$. Компонента
поляризации, ответственная за генерацию второй гармоники, согласно
(5), имеет вид\begin{align}
\varepsilon_0\chi_2E^2 = \frac{\varepsilon_0 \chi_2 E^2_{0\omega}}{2}cos\left[2\left(\omega t - k_\omega z\right)\right]
\end{align}Это выражение описывает поляризацию среды, осциллирующую на частоте
$2 \omega$ и распространяющуюся в среде в виде волны.

Данная волна поляризации излучает световую волну на частоте $2 \omega$,
электрическое поле которой запишется следующим образом:\begin{align}
E_{2\omega} = E_{02\omega}cos(2\omega t - k_{2\omega} z)
\end{align} Если $\overline{n}_\omega$ и $\overline{n}_{2\omega}$ - фазовые показатели преломления среды на частотах ω и
2 ω , то модули волновых векторов основной волны и волны второй гар-
моники будут иметь вид:\begin{align}
k_\omega = \frac{\omega \overline{n}_\omega}{c};\ k_{2\omega} = \frac{{2\omega} \overline{n}_{2\omega}}{c},
\end{align} где с – скорость света в вакууме. Таким образом, световая волна второй гармоники получает
энергию от излучения на основной частоте волны через волну поляризации. Очевидно, передача энергии будет происходить с наибольшей эффективностью, если разность фаз между волнами (6) и (7) будет сохраняться постоянной на достаточно больших расстояниях. В действительности же из-за дисперсии показателя преломления фазовые скорости волн будут различны. Поэтому при прохождении в среде
расстояния l между волнами (6) и (7) возникнет сдвиг фаз:\begin{align}
\Delta \varphi = (k_{2\omega} - 2k_\omega)l
\end{align} Сдвиг фаз достигнет значения $\Delta \varphi = \pi$  по прохождении волнами расстояния, равного:\begin{align}
l_c = \frac{\pi}{k_{2\omega} - 2k_\omega} = \frac{\lambda_\omega}{4(\overline{n}_{2\omega}-\overline{n}_\omega)},
\end{align} где $\lambda_\omega$ – длина волны излучения на основной частоте, а значение $l_c$ называется когерентной длиной. Очевидно, накопления энергии волной
второй гармоники нельзя ожидать на расстояниях, больших $l_c$. 

Условие \begin{align}
\overline{n}_{2\omega} = \overline{n}_\omega
\end{align} называется условием фазового или волнового синхронизма. При его выполнении все точки среды преобразуют излучение, т.е. их вклады в интенсивность света на удвоенной частоте складываются. В противном случае начиная с какого-то расстояния осциллирующие на двойной частоте диполи начиная с некоторого расстояния начинают работать в противофазе, мешая друг другу.


Но как добиться выполнения условия фазового синхронизма? Вероятность того, что нам удастся найти нелинейную среду, у которой для данной частоты коэффициент преломления $\overline{n}_{2\omega}$ окажется равным $\overline{n}_\omega$, равна нулю. Изящный способ решения этой проблемы придумали в 1962 году Дж. Джолдэйн и Р. Терхьюн. Они предложили использовать для удвоения частоты анизотропные кристаллы. В этом случае в среде распространяется два луча – обыкновенный и необыкновенный. Коэффициент преломления для обыкновенного луча независит от направления его волнового вектора, в то время как коэффициент преломления обыкновенного луча зависит от направления его распространения.


	В области нормальной дисперсии коэффициент преломления увеличивается с ростом частоты. Если при этом для необыкновенной волны коэффициент преломления уменьшается по мере увеличения угла между волновым вектором и оптической осью, то вполне может оказаться, что существует такое направление, для которого коэффициент преломления для необыкновенного луча на двойной частоте и для обыкновенного – на одинарной окажутся одинаковыми.
	
	
	Выполнение условия фазового синхронизма удается получить просто вращая такой двулучепреломляющий, нелинейный кристалл.


Таким образом, основная проблема генерации излучения второй гармоники
заключается в том, что если рассматривать изотропную среду, в силу
наличия дисперсии, между падающей волной и волной удвоенной частоты
будет постоянно меняться фаза, что приведет к невозможности накопления
энергии во второй гармонике. Для избежания этого эффекта (выполнения так
называеого \emph{условия синхронизма}) были применены
двулучепреломляющие пластинки и волны с разной поляризацией.\subsection{Результаты}В нашем эксперименте была успешно получена вторая гармоника с длиной
волны $\approx532$ нм, зеленая. Ниже её спектр и форма импульса. 
\begin{figure}[h]
\includegraphics[width=0.5\textwidth]{LEMPH Report_files/green laser.jpg}
\caption{Спектр второй гармоники}
\end{figure}
\begin{figure}
    \includegraphics[width=0.5\textwidth]{LEMPH Report_files/without filters.jpg}
	\caption{Осцилограмма второй гармоники}
    \end{figure}
\newpage

\section{Лазер на красителях}Особый класс лазерных активных сред представляют жидкие растворы многих органических красителей, которые имеют интенсивные
и широкие спектры флуоресценции при возбуждении в видимой и ультрафиолетовой областях. 

В нашей работе в качестве красителя использовался родамин 6G.

На нем удалось получить генерацию на длине волны примерно 598 нм. Были сняты зависимости спектра излучения от напряжения накачки и от угла поворота призмы с красителем. Но теоретическая перестройка длины волны генерации не была обнаружена из-за недостаточной чувствительности спектрометра. Разрешающая способность спектрометра проиллюстрирована на соотвествующем рисунке в разделе "Люминисценция".
\begin{figure}[h]
\centering
\includegraphics[width=0.5\textwidth]{LEMPH Report_files/scheme.jpg}
\caption{Схема установки \label{fig:kakaha}}
\end{figure}
\begin{figure}[h]
\centering
\includegraphics[width=0.5\textwidth]{LEMPH Report_files/1,2.jpg}
\caption{Спектр генерации при напряжении накачки 1.2 кВ}
\end{figure}
\begin{figure}[h]
\centering
\includegraphics[width=0.5\textwidth]{LEMPH Report_files/1,5.jpg}
\caption{Спектр генерации при напряжении накачки 1.5 кВ}
\end{figure}
\begin{figure}[h]
\includegraphics[width=0.5\textwidth]{LEMPH Report_files/lumin and gener.jpg}
\caption{Спектр генерации на красителе, наложенный на спектр люминисценции красителя}
\end{figure}

Видно, что была получена действительно генерация, т.к. полученный пик гораздо уже пика люминисценции.


\begin{figure}[h!]
    \includegraphics[width=0.5\textwidth]{LEMPH Report_files/with filters.jpg}
    \caption{Форма импульса лазера на красителях}
    \end{figure}
        

        \renewcommand{\indexname}{Index}
  

    % End of document
    \end{document}


